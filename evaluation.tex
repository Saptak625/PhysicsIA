\section{Conclusion}
Therefore, according to this experiment, there is a very strong correlation ($R^2=0.9988$) for a negative exponential relationship between the resistance and the temperature of an NTC Thermistor.
This matches my hypothesis for this experiment. Furthermore, the individual trial percent uncertainties were far below 5\%, which means that our measurement system is precise. However, some of the average percent uncertainties for the resistances at low temperatures were above 5\%, which means that the experiment had a bad procedure for these levels. Finally, since the uncertainty for the slope $\beta$ was less than 5\% and greater than the percent error for $\beta$, the final experiment results were precise and accurate.

\section{Sources for Error}
Error could have been introduced from 4 sources: measurement of temperature using the Vernier Temperature Probe, measurement of resistance using the Multimeter, data collection using video analysis, and the indirect heating of the thermistor using a water medium.

First, the measurement of temperature using the Vernier Temperature Probe contributes to random error since the device uncertainty for the probe is $\pm\;0.1^\circ C$. This random error from this source is negligible in this experiment. Next, the measurement of resistance using the Multimeter also contributes to random error since the device uncertainty for the device is $\pm\;0.01\;k\Omega$. This random error from this source is negligible at high temperatures, but at lower temperatures ($\le 35.0^\circ C$) the percent uncertainties exceeded 5\% indicating a significant random error. Then, since all temperature and resistance measurements were derived through video analysis, some random error was introduced. However, the accurate scrubbing on high frame rate and high-resolution videos utilized in the analysis significantly mitigated random error over collecting the data by eye. Therefore, the random error from this source is negligible in this experiment. Finally, since the thermistor was indirectly heated through a water medium, the temperature of the water could have been higher than that of the thermistor leading to systematic error. However, since the heating was put on the lowest setting, the temperature increased evenly and steadily, hence minimizing this error.

In conclusion, most sources of error were minimal throughout the experiment. However, the greater than 5\% percent uncertainties for resistance measurements at low temperatures due to random error were significant.

\section{Strengths}
One strength of this experiment is the use of video analysis to collect accurate temperature and resistance measurements. As the temperature steadily and evenly rose while being heated, video analysis was much more accurate than manually collecting the data by eye during the experiment. This also enabled me to take more frequent measurements, which helped me create more levels and hence data points for my regressions. Another strength of this experiment was the slow cooling procedure. After each trial, all equipment (beaker, temperature probe, and thermistor) was gradually cooled down back to room temperature. This ensured that the equipment did not experience a rapid temperature change, which could wear it down or damage it leading to varying results. Thus, in each trial, all equipment started in the same state as the previous trial. Finally, the slow heating utilized during the experiment was a major strength of this experiment. The slow heating ensured that the temperature rose steadily and evenly allowing for accurate and precise data collection through the video analysis. Furthermore, the slow heating mitigated systematic errors due to the differences in temperatures of the water, temperature probe, and thermistor.

\section{Weaknesses}
One weakness of this experiment is the loose connections between the circuit elements leading to variable results. Since alligator clips were utilized to connect the leads of the Multimeter to the NTC Thermistor, slightly loose values would cause significant variations in resistance readings. Furthermore, another weakness of this experiment is the NTC Thermistor utilized in this experiment. This thermistor had low accuracy at lower temperatures creating high random error, which led to the high average percent uncertainties for temperatures below $35.0^\circ C$. These were major experiment design flaws that need to be revised for future trials.

\section{Improvements}
There are multiple improvements that could be applied to this experiment to address its weaknesses. First, fewer connections in the circuit could be utilized to reduce the effect of loose connections. Specifically, the Multimeter could be directly attached to the NTC thermistor rather than via alligator clips. Second, an NTC thermistor with better accuracy at lower temperatures ($\le 35.0^\circ C$) could be used. This would reduce the high random error and hence the high average percent uncertainty for these temperatures. These improvements would greatly improve both the accuracy and precision of this experiment.

\section{Extensions}
There are multiple extensions to this experiment that might be worthy of future study. First, as mentioned in the Section \ref{section:introduction} \nameref{section:introduction}, this experiment could be extended to test the relationship between the resistance and temperature of a superconductor. Specifically, It would be interesting to see the change in this relationship prior to and after the critical temperature $T_c$ of a superconductive material such as YBCO. This experiment could be conducted in a similar fashion to this experiment, except, instead of using a range of temperatures above room temperature, liquid nitrogen could be used to achieve a range below and above the $T_c$ of a YBCO at 93 K \citep{butera1997dependence}.
Another experiment could explore other models to describe the relationship between the resistance and temperature of a thermistor. In industry, the Steinhart-Hart Model, an extension of the Beta Model used in this experiment, is used to describe this relationship. The following  is the equation for the Steinhart-Hart Model:
\[\frac{1}{T}=A+B\ln{R}+C[\ln{R}]^3\]
The procedure for the experiment would be similar to this experiment, but rather than using an exponential fit and the Beta Model to linearize my data, a cubic polynomial fit and the Steinhart-Hart Model would be utilized. This will improve the accuracy of the regression even more as the Beta Model simplifies the Steinhart-Hart Model by removing its cubic term \citep{rana2018fpga}.
